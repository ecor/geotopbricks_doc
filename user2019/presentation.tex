\documentclass[ignorenonframetext,]{beamer}
\setbeamertemplate{caption}[numbered]
\setbeamertemplate{caption label separator}{: }
\setbeamercolor{caption name}{fg=normal text.fg}
\beamertemplatenavigationsymbolsempty
\usepackage{lmodern}
\usepackage{amssymb,amsmath}
\usepackage{ifxetex,ifluatex}
\usepackage{fixltx2e} % provides \textsubscript
\ifnum 0\ifxetex 1\fi\ifluatex 1\fi=0 % if pdftex
  \usepackage[T1]{fontenc}
  \usepackage[utf8]{inputenc}
\else % if luatex or xelatex
  \ifxetex
    \usepackage{mathspec}
  \else
    \usepackage{fontspec}
  \fi
  \defaultfontfeatures{Ligatures=TeX,Scale=MatchLowercase}
\fi
% use upquote if available, for straight quotes in verbatim environments
\IfFileExists{upquote.sty}{\usepackage{upquote}}{}
% use microtype if available
\IfFileExists{microtype.sty}{%
\usepackage{microtype}
\UseMicrotypeSet[protrusion]{basicmath} % disable protrusion for tt fonts
}{}
\newif\ifbibliography
\hypersetup{
            pdftitle={geotopbricks},
            pdfauthor={Emanuele Cordano, Giacomo Bertoldi, Elisa Bortoli},
            pdfborder={0 0 0},
            breaklinks=true}
\urlstyle{same}  % don't use monospace font for urls
\usepackage{color}
\usepackage{fancyvrb}
\newcommand{\VerbBar}{|}
\newcommand{\VERB}{\Verb[commandchars=\\\{\}]}
\DefineVerbatimEnvironment{Highlighting}{Verbatim}{commandchars=\\\{\}}
% Add ',fontsize=\small' for more characters per line
\usepackage{framed}
\definecolor{shadecolor}{RGB}{248,248,248}
\newenvironment{Shaded}{\begin{snugshade}}{\end{snugshade}}
\newcommand{\KeywordTok}[1]{\textcolor[rgb]{0.13,0.29,0.53}{\textbf{#1}}}
\newcommand{\DataTypeTok}[1]{\textcolor[rgb]{0.13,0.29,0.53}{#1}}
\newcommand{\DecValTok}[1]{\textcolor[rgb]{0.00,0.00,0.81}{#1}}
\newcommand{\BaseNTok}[1]{\textcolor[rgb]{0.00,0.00,0.81}{#1}}
\newcommand{\FloatTok}[1]{\textcolor[rgb]{0.00,0.00,0.81}{#1}}
\newcommand{\ConstantTok}[1]{\textcolor[rgb]{0.00,0.00,0.00}{#1}}
\newcommand{\CharTok}[1]{\textcolor[rgb]{0.31,0.60,0.02}{#1}}
\newcommand{\SpecialCharTok}[1]{\textcolor[rgb]{0.00,0.00,0.00}{#1}}
\newcommand{\StringTok}[1]{\textcolor[rgb]{0.31,0.60,0.02}{#1}}
\newcommand{\VerbatimStringTok}[1]{\textcolor[rgb]{0.31,0.60,0.02}{#1}}
\newcommand{\SpecialStringTok}[1]{\textcolor[rgb]{0.31,0.60,0.02}{#1}}
\newcommand{\ImportTok}[1]{#1}
\newcommand{\CommentTok}[1]{\textcolor[rgb]{0.56,0.35,0.01}{\textit{#1}}}
\newcommand{\DocumentationTok}[1]{\textcolor[rgb]{0.56,0.35,0.01}{\textbf{\textit{#1}}}}
\newcommand{\AnnotationTok}[1]{\textcolor[rgb]{0.56,0.35,0.01}{\textbf{\textit{#1}}}}
\newcommand{\CommentVarTok}[1]{\textcolor[rgb]{0.56,0.35,0.01}{\textbf{\textit{#1}}}}
\newcommand{\OtherTok}[1]{\textcolor[rgb]{0.56,0.35,0.01}{#1}}
\newcommand{\FunctionTok}[1]{\textcolor[rgb]{0.00,0.00,0.00}{#1}}
\newcommand{\VariableTok}[1]{\textcolor[rgb]{0.00,0.00,0.00}{#1}}
\newcommand{\ControlFlowTok}[1]{\textcolor[rgb]{0.13,0.29,0.53}{\textbf{#1}}}
\newcommand{\OperatorTok}[1]{\textcolor[rgb]{0.81,0.36,0.00}{\textbf{#1}}}
\newcommand{\BuiltInTok}[1]{#1}
\newcommand{\ExtensionTok}[1]{#1}
\newcommand{\PreprocessorTok}[1]{\textcolor[rgb]{0.56,0.35,0.01}{\textit{#1}}}
\newcommand{\AttributeTok}[1]{\textcolor[rgb]{0.77,0.63,0.00}{#1}}
\newcommand{\RegionMarkerTok}[1]{#1}
\newcommand{\InformationTok}[1]{\textcolor[rgb]{0.56,0.35,0.01}{\textbf{\textit{#1}}}}
\newcommand{\WarningTok}[1]{\textcolor[rgb]{0.56,0.35,0.01}{\textbf{\textit{#1}}}}
\newcommand{\AlertTok}[1]{\textcolor[rgb]{0.94,0.16,0.16}{#1}}
\newcommand{\ErrorTok}[1]{\textcolor[rgb]{0.64,0.00,0.00}{\textbf{#1}}}
\newcommand{\NormalTok}[1]{#1}
\usepackage{graphicx,grffile}
\makeatletter
\def\maxwidth{\ifdim\Gin@nat@width>\linewidth\linewidth\else\Gin@nat@width\fi}
\def\maxheight{\ifdim\Gin@nat@height>\textheight0.8\textheight\else\Gin@nat@height\fi}
\makeatother
% Scale images if necessary, so that they will not overflow the page
% margins by default, and it is still possible to overwrite the defaults
% using explicit options in \includegraphics[width, height, ...]{}
\setkeys{Gin}{width=\maxwidth,height=\maxheight,keepaspectratio}

% Prevent slide breaks in the middle of a paragraph:
\widowpenalties 1 10000
\raggedbottom

\AtBeginPart{
  \let\insertpartnumber\relax
  \let\partname\relax
  \frame{\partpage}
}
\AtBeginSection{
  \ifbibliography
  \else
    \let\insertsectionnumber\relax
    \let\sectionname\relax
    \frame{\sectionpage}
  \fi
}
\AtBeginSubsection{
  \let\insertsubsectionnumber\relax
  \let\subsectionname\relax
  \frame{\subsectionpage}
}

\setlength{\parindent}{0pt}
\setlength{\parskip}{6pt plus 2pt minus 1pt}
\setlength{\emergencystretch}{3em}  % prevent overfull lines
\providecommand{\tightlist}{%
  \setlength{\itemsep}{0pt}\setlength{\parskip}{0pt}}
\setcounter{secnumdepth}{0}
%% EMOS rmarkdown/beamer header
%% Mark van der Loo (2016)
\usepackage{listings}
\usepackage{mdframed}
\usepackage{tikz}

\usetikzlibrary{arrows, positioning, decorations.pathreplacing} 
\usepackage{pgfplots}
\usepackage{color}


% CBS corporate light blue
\definecolor{corplightblue}{HTML}{00a1cd}
% CBS corporate dark blue
\definecolor{corpdarkblue}{HTML}{0058b8}

% set title colors
\setbeamercolor{frametitle}{fg=corpdarkblue}
\setbeamercolor{title}{fg=corpdarkblue}
\setbeamercolor{block title}{fg=corpdarkblue}

\setbeamerfont{normal text}{parent=structure}
% nicer, rounder font for title
\setbeamerfont{title}{series=\bfseries,parent=structure}
% frame titles in boldface
\setbeamerfont{frametitle}{series=\bfseries}
% block title font
\setbeamerfont{block title}{parent=structure}

% enumeration
\setbeamertemplate{itemize item}{\color{corpdarkblue}$\blacktriangleright$}
\setbeamertemplate{itemize subitem}{\color{corpdarkblue}$-$}
\setbeamercolor*{enumerate item}{fg=corpdarkblue}
\setbeamercolor*{enumerate subitem}{fg=corpdarkblue}
\setbeamercolor*{enumerate subsubitem}{fg=corpdarkblue}


\makeatletter
\newcommand\HUGE{\@setfontsize\Huge{50}{60}}
\makeatother    


% nicer, rounder font for title

\setbeamertemplate{title page}{
\begin{picture}(0,0)
\put(0,70){\usebeamerfont{title}{\textcolor{corpdarkblue}{\inserttitle}\par}}
\put(0,55){\usebeamerfont{subtitle}{\textcolor{corpdarkblue}{\insertsubtitle}\par}}
\put(0,30){
\small Emanuele Cordano (Rendena100) 
}
\put(0,10){
\texttt{\small github.com/ecor}
}
\put(0,-10){
\small Giacomo Bertoldi , Elisa Bortoli (EURAC Ecohydro)
}
\put(0,-30){
\texttt{\small github.com/Ecohydro}
}

\put(150,-100){
\includegraphics[height=3cm]{resources/logo/logo_geotop.jpeg}
}
\put(250,-100){
\includegraphics[height=3cm]{resources/logo/logo_useR2019.png}
}
\end{picture}

}



\setbeamertemplate{frametitle}
{
  \begin{beamercolorbox}{frametitle}
  \vskip2.7ex\insertframetitle
  \end{beamercolorbox}
}


% remove the space-eating navigation symbols
\beamertemplatenavigationsymbolsempty

% convenience function to define rgb colors in tikz
\tikzset{xcolor/.code args={#1=#2}{
     \definecolor{mytemp}{rgb}{#2}
     \tikzset{#1=mytemp}
  }
}


\usebackgroundtemplate{
   \begin{picture}(0,0)
    \put(270, -273){%
      \raisebox{5mm}{\tiny useR2019, Toulouse,France }\includegraphics[height=8mm]{resources/logo/logo_useR2019.png}
    }
\put(0,-273){
\includegraphics[height=7mm]{resources/logo/logo_geotop.jpeg}
\includegraphics[height=7mm]{resources/logo/logo_eurac.png}
\includegraphics[height=7mm]{resources/logo/logo_rendena100_textoutside_small.jpg}
}
\end{picture}
}


\newcommand{\la}[1]{\boldsymbol{#1}}


\def\begincols{\begin{columns}}
\def\begincol{\begin{column}}
\def\endcol{\end{column}}
\def\endcols{\end{columns}}

\title{geotopbricks}
\subtitle{An R Package for the Distributed Hydrological Model GEOtop}
\author{Emanuele Cordano, Giacomo Bertoldi, Elisa Bortoli}
\date{github/ecor / useR2019}

\begin{document}
\frame{\titlepage}

\begin{frame}{Who are we?}

\begincols

\begincol{0.28\textwidth}

\includegraphics[width=1.00000\textwidth]{resources/images/emanuele.jpg}~

\includegraphics[width=0.70000\textwidth]{resources/images/giacomo.jpg}\\
\includegraphics[width=0.70000\textwidth]{resources/images/elisa.jpg}\\
\endcol

\begincol{0.7\textwidth}

\begin{itemize}
\tightlist
\item
  Environmental engineers with hydrological background (more
  deterministic and physically-based than statics!)
\item
  Some of us are researchers, other are self-employed and freelancers -
  www.rendena100.eu . - Some of us are authors of several R-packages and
  R enthusiast.
\item
  Some of us are developers of GEOtop hydrologic models with skills in
  hydrology, environmental science and also in C/C++, parallell
  programming, High Perfomance Computing, etc.
\end{itemize}

\endcol

\endcols

\end{frame}

\begin{frame}{Hydrology}

Scientific study of the movement, distribution, and quality of water,
including the water cycle, water resources and environmental watershed
sustainability.{[}\emph{Wikipedia}{]} \begincols
\begincol{.48\textwidth}
\includegraphics[width=1.00000\textwidth]{resources/images/geotop_landscape.png}\\
\endcol
\begincol{.48\textwidth}
\includegraphics[width=1.00000\textwidth]{resources/images/valdifumo.jpg}\\
\endcol
\endcols

\end{frame}

\begin{frame}{Hydrological Models}

\begincols
 \begincol{.58\textwidth}

\includegraphics[width=0.70000\textwidth]{resources/images/geotop_grid_mod2.jpg}~
\includegraphics[width=0.50000\textwidth]{resources/images/water_balance.png}\\
\endcol
\begincol{.02\textwidth} \endcol
\begincol{.36\textwidth}

Models that estimate water river discharge, soil water
content,evapo-transpiration, etc. (\emph{output}) in function of weather
forcings and soil/land/geomorphological characterization (\emph{input}).

\endcol
\endcols
Soil water mass balance equation:
\(\frac{\partial \theta}{\partial t} = \nabla \cdot \left[ K \left(\nabla (\psi+z_f) \right) \right] +S\)

Soil Heat (energy) balance equation:
\(C_s \frac{\partial T_s}{\partial t} = \nabla \cdot \left[ K_t ( \nabla T_s ) \right] +\lambda S\)

\end{frame}

\begin{frame}{GEOtop Hydrological Model}

GEOtop hydrological model solves water mass balance and energy balance
equations coupled with the exchanges between terrain and lower atmoshere
in the following two setup configurations:

\begin{itemize}
\item
  \textbf{1D}: only vertical fluxes \(\,\to\,\) balances at local scale
  (only in one soil column)
\item
  \textbf{3D}: vertical and lateral fluxes \(\,\to\,\) balances at basin
  scale
\end{itemize}

\includegraphics[width=0.60000\textwidth]{resources/images/geotop_ET_SWC.png}\\

\end{frame}

\begin{frame}{GEOtop Hydrological Model Software Package / Source Code}

GEOtop Hydrological Model is an open source software package (GPL3
licence):

\begin{itemize}
\tightlist
\item
  written in C/C++
\item
  released in 2014 (version 2.0) as free open-source project, a
  re-engineering process is going to finish (version 3.0);
\item
  scientifically tested and published;
\end{itemize}

Source code and documentation are available on GitHub repository:
\url{http://geotopmodel.github.io/geotop/}.

\includegraphics[width=0.90000\textwidth]{resources/images/geotop_paper_2017.png}\\

\end{frame}

\begin{frame}{geotopbricks R Package: Why?}

\begin{itemize}
\tightlist
\item
  complexity in input/output/configuration files (\emph{``frontend''})
  and data difficult to handle
\item
  need of user friendly environment for to GEOtop data tidying and data
  analytics (e.g. \emph{R})
  \includegraphics[width=0.90000\textwidth]{resources/images/Capture_IO_GEOtopJPG.JPG}\\
\end{itemize}

\end{frame}

\begin{frame}[fragile]{GEOtop Simulation Configuration File
(geotop.inpts)}

GEOtop simulation is a directory containing a configuration file, called
\textbf{geotop.inpts} filled with a keywords system addressing to
simulation options (e.g.~simulation period); \textbf{input files}
(e.g.~meteorological forcings, soil and geomorphology of the basin);
\textbf{output files} (spatio-temporal maps - raster and time series -
of the results).

\begin{verbatim}
InitDateDDMMYYYYhhmm=09/04/2014 18:00  
EndDateDDMMYYYYhhmm =01/01/2016 00:00 
[...] 
MeteoFile           ="meteoB2_irr" 
PointOutputFile     ="tabs/point" 
\end{verbatim}

\end{frame}

\begin{frame}{geotopbricks R Package: What it Does}

The aim of \textbf{geotopbricks} , starting in 2013, is to import all
GEOtop simulaton data into the \textbf{R} environment by using the
\emph{keyword-value} syntax of \emph{geotop.inpts}.
\textbf{geotopbricks} does the following actions:

\begin{itemize}
\tightlist
\item
  parsing \emph{geotop.inpts} configuration file;
\item
  deriving from \emph{geotop.inpts}'s keywords the source files of I/O
  data;
\item
  importing time series (e.g.~precipitation, temperature, soil water
  content, snow) as \emph{zoo} or \emph{data.frame} objects;
\item
  importing spatially and spatio-temporal gridded objects as
  \emph{RasterLayer-class} or \emph{RasterBrick-class} objects
  (\textbf{raster} package).
\end{itemize}

\end{frame}

\begin{frame}{1D GEOtop Simulation in an Alpine Site: 2 Points}

Estimation of soil water content (SWC) in two points \textbf{P2} and
\textbf{B2} located in Val Mazia/Matsch, South Tyrol, Italy
\url{http://lter.eurac.edu/en}. \begincols

\begincol{0.30\textwidth}
\includegraphics[width=1.00000\textwidth]{resources/images/B2_P2.png}\\
\includegraphics[width=1.00000\textwidth]{resources/images/mazia_2.png}\\
\endcol
\begincol{0.30\textwidth}
\includegraphics[width=1.00000\textwidth]{resources/images/water_balance}\\
\endcol
\begincol{0.40\textwidth}

\includegraphics[width=1.00000\textwidth]{resources/images/B2_new.png}\\
\endcol
 \endcols

\end{frame}

\begin{frame}[fragile]{1D GEOtop Simulation in an Alpine Site: B2}

Here is the directory containing files of B2 point simulation:

\begin{Shaded}
\begin{Highlighting}[]
\KeywordTok{library}\NormalTok{(geotopbricks) }

\NormalTok{## SET GEOTOP SIMULATION DIRECTORY}
\NormalTok{wpath_B2 <-}\StringTok{ "resources/simulation/Matsch_B2_Ref_007"} 
\end{Highlighting}
\end{Shaded}

\includegraphics[width=1.00000\textwidth]{resources/images/geotop_folder_B2.png}\\

\end{frame}

\begin{frame}[fragile]{Getting Simulation Input Data}

Meteorological forcings time series are imported and saved as
\textbf{meteo} variable (class \textbf{zoo}). This variable is retrieved
through the GEOtop keyword \textbf{MeteoFile} :

\begin{Shaded}
\begin{Highlighting}[]
\NormalTok{tz <-}\StringTok{ "Etc/GMT-1"}
\NormalTok{meteo <-}\StringTok{ }\KeywordTok{get.geotop.inpts.keyword.value}\NormalTok{(}
  \StringTok{"MeteoFile"}\NormalTok{,}
  \DataTypeTok{wpath=}\NormalTok{wpath_B2,}
  \DataTypeTok{data.frame=}\OtherTok{TRUE}\NormalTok{,}
  \DataTypeTok{tz=}\NormalTok{tz)}
\KeywordTok{class}\NormalTok{(meteo)}
\end{Highlighting}
\end{Shaded}

\begin{verbatim}
## [1] "zoo"
\end{verbatim}

\end{frame}

\begin{frame}[fragile]{Getting Simulation Input Data (verify)}

Meteorological time series once imported can be printed:

\begin{Shaded}
\begin{Highlighting}[]
\KeywordTok{head}\NormalTok{(meteo[}\DecValTok{12}\OperatorTok{:}\DecValTok{14}\NormalTok{,}\KeywordTok{c}\NormalTok{(}\StringTok{"Iprec"}\NormalTok{,}\StringTok{"AirT"}\NormalTok{,}\StringTok{"Swglobal"}\NormalTok{)])}
\end{Highlighting}
\end{Shaded}

\begin{verbatim}
##                     Iprec  AirT Swglobal
## 2009-10-02 11:00:00     0 12.38   396.02
## 2009-10-02 12:00:00     0 13.12   500.07
## 2009-10-02 13:00:00     0 13.96   564.02
\end{verbatim}

\begin{Shaded}
\begin{Highlighting}[]
\KeywordTok{head}\NormalTok{(meteo[}\DecValTok{12}\OperatorTok{:}\DecValTok{14}\NormalTok{,}\KeywordTok{c}\NormalTok{(}\StringTok{"RelHum"}\NormalTok{,}\StringTok{"WindSp"}\NormalTok{,}\StringTok{"WindDir"}\NormalTok{)])}
\end{Highlighting}
\end{Shaded}

\end{frame}

\begin{frame}{Precipitation and Air Temperature at B2}

\includegraphics{presentation_files/figure-beamer/unnamed-chunk-5-1.pdf}

\end{frame}

\begin{frame}[fragile]{Getting Simulation Output Data}

Soil Water Content Profile:

\begin{Shaded}
\begin{Highlighting}[]
\NormalTok{tz <-}\StringTok{ "Etc/GMT-1"}
\NormalTok{SWC_B2  <-}\StringTok{ }\KeywordTok{get.geotop.inpts.keyword.value}\NormalTok{(}
  \StringTok{"SoilLiqContentProfileFile"}\NormalTok{,}
  \DataTypeTok{wpath =}\NormalTok{ wpath_B2,}
  \DataTypeTok{data.frame =} \OtherTok{TRUE}\NormalTok{,}
  \DataTypeTok{date_field =} \StringTok{"Date12.DDMMYYYYhhmm."}\NormalTok{,}
  \DataTypeTok{tz =}\NormalTok{ tz,}
  \DataTypeTok{zlayer.formatter =} \StringTok{"z%04d"}
\NormalTok{)}
\KeywordTok{help}\NormalTok{(get.geotop.inpts.keyword.value) ## for more details!}
\end{Highlighting}
\end{Shaded}

\end{frame}

\begin{frame}[fragile]{Getting Simulation Output Data (at P2)}

Analogously for P2:

\begin{Shaded}
\begin{Highlighting}[]
\NormalTok{wpath_P2 <-}\StringTok{ "resources/simulation/Matsch_P2_Ref_007"} 
\NormalTok{SWC_P2  <-}\StringTok{ }\KeywordTok{get.geotop.inpts.keyword.value}\NormalTok{(}
  \StringTok{"SoilLiqContentProfileFile"}\NormalTok{,}
  \DataTypeTok{wpath =}\NormalTok{ wpath_P2,}
  \DataTypeTok{data.frame =} \OtherTok{TRUE}\NormalTok{,}
  \DataTypeTok{date_field =} \StringTok{"Date12.DDMMYYYYhhmm."}\NormalTok{,}
  \DataTypeTok{tz =} \StringTok{"Etc/GMT-1"}\NormalTok{,}
  \DataTypeTok{zlayer.formatter =} \StringTok{"z%04d"}\NormalTok{)}
\end{Highlighting}
\end{Shaded}

\end{frame}

\begin{frame}{Soil Water Content at P2 and B2}

\includegraphics{presentation_files/figure-beamer/unnamed-chunk-10-1.pdf}

\end{frame}

\begin{frame}[fragile]{3D Spatially Distributed Simulation: Val
Venosta/Vinschgau - Upper Adige River Basin - Alps - I/CH/A}

\begin{Shaded}
\begin{Highlighting}[]
\NormalTok{wpath_3D <-}\StringTok{ 'resources/simulation/Vinschgau'}
\NormalTok{basin <-}\StringTok{ }\KeywordTok{get.geotop.inpts.keyword.value}\NormalTok{(}\StringTok{"LandCoverMapFile"}\NormalTok{,}
              \DataTypeTok{wpath=}\NormalTok{wpath_3D,}\DataTypeTok{raster=}\OtherTok{TRUE}\NormalTok{)}
\NormalTok{basin}
\end{Highlighting}
\end{Shaded}

\begin{verbatim}
## class       : RasterLayer 
## dimensions  : 48, 63, 3024  (nrow, ncol, ncell)
## resolution  : 1000, 1000  (x, y)
## extent      : 598000, 661000, 5145000, 5193000  (xmin, xmax, ymin, ymax)
## coord. ref. : +proj=utm +zone=32 +ellps=WGS84 +datum=WGS84 +units=m +no_defs +towgs84=0,0,0 
## data source : in memory
## names       : layer 
## values      : 1, 11  (min, max)
\end{verbatim}

\end{frame}

\begin{frame}{Input GeoSpatial Map: Elevation and Weather Station}

\includegraphics{presentation_files/figure-beamer/unnamed-chunk-12-1.pdf}

\end{frame}

\begin{frame}[fragile]{3D Spatially Distributed Simulation (Output
Geospatial Map): Soil Water Content}

\includegraphics{presentation_files/figure-beamer/unnamed-chunk-13-1.pdf}

\begin{verbatim}
brickFromOutputSoil3DTensor("SoilLiqContentTensorFile", 
wpath=wpath_3D,when="2011-08-16 12:00:00 +01")
\end{verbatim}

\end{frame}

\begin{frame}{3D Spatially Distributed Simulation (Output Geospatial
Map): Surface Water Discharge at the Outlet}

\includegraphics{presentation_files/figure-beamer/unnamed-chunk-15-1.pdf}

\end{frame}

\begin{frame}{Discussion}

\begin{itemize}
\item
  \textbf{geotopbricks} allows graphical representation using R of
  GEOtop results , useful for hydrologigists and reaserchers;
\item
  Through \textbf{geotopbricks} user can interact between R and GEOtop
  using R enviroment and GEOtop keywords system, without using the
  GEOtop simulation structure.
\item
  Processing of a GEOtop simulation is always reproducible for any other
  simulation; results can be automatically documented in reports or
  presentations.
\end{itemize}

\end{frame}

\begin{frame}{Conclusions and Way Forward}

\begin{itemize}
\item
  \textbf{geotopbricks} is an interface of GEOtop in R speaking the
  language of GEOtop;
\item
  R code based on \textbf{geotopbricks} can help the implementation of
  further package or apps: analityics, model calibration, visualization.
\item
  Open Source (and not only) Hydrological Model needs powerful
  interfaces to process I/O in a FAIR way;
\end{itemize}

\includegraphics[width=0.75000\textwidth]{resources/images/FAIR_data_principles.jpg}\\

\end{frame}

\begin{frame}{Finally}

Aknowledgements to

\begin{itemize}
\tightlist
\item
  all \textbf{GEOtop} developers and users' group, in particular
  \textbf{Matteo Dall'Amico, Stefano Cozzini, Alberto Sartori, Stefano
  Endrizzi, Samuel Senoner, Riccardo Rigon}, who provided images about
  GEOtop for this presentation
\item
  the community of \textbf{R} whose packages allow to analize and
  visualise GEOtop data.
\end{itemize}

If intertested? See and follow us on (www.geotop.org) or
(\url{https://cran.r-project.org/package=geotopbricks})

Thank you for your attention! / Merci pour votre attention!\\
Find us as \textbf{@ecor} (presenter) or \textbf{@EURAC-Ecohydro}
(co-authors) on \emph{GitHub}.

\end{frame}

\begin{frame}{Addendum}

\end{frame}

\begin{frame}{GEOtop Hydrological Model Flowchart}

\begincols

\begincol{.69\textwidth}
\includegraphics[width=1.00000\textwidth]{resources/images/geotop_revised.png}\\
\endcol

\begincol{.30\textwidth}

\begin{itemize}
\item
  \textbf{Input}: meteo data, elevations, soil parameters,\ldots{}
\item
  \textbf{Output}: snow cover, soil temperature, soil moisture,\ldots{}
\end{itemize}

\endcol
\endcols

\end{frame}

\begin{frame}{Soil Water Pressure Head at P2 and B2}

\includegraphics{presentation_files/figure-beamer/unnamed-chunk-16-1.pdf}

\end{frame}

\begin{frame}{Example of an Output Data Analytics (Soil Moisture
Distribution)}

Distribution of daily aggregated soil water contant at a 18 cm depth:
\includegraphics{presentation_files/figure-beamer/unnamed-chunk-17-1.pdf}

More details on the
\href{https://github.com/ecor/geotopbricks_doc/blob/master/erum2018_poster/erum2018_poster_cordano_et_al.png}{\textbf{eRum2018}
poster}.

\end{frame}

\end{document}
